\ifx\allfiles\undefined
\documentlecture[12pt, a4paper, oneside, UTF8]{ctexbook}  %  这一句是新增加的
\usepackage[dvipsnames]{xcolor}
\usepackage{amsmath}   % 数学公式
\usepackage{graphicx}
\usetikzlibrary{arrows, calc, decorations.pathmorphing}
\allowdisplaybreaks % 允许公式跨页换行
\newcommand{\pa}{\partial}
\newcommand{\mathminus}{\!\!-\!\!} % 数学环境连字符
\newcommand{\vsup}[1]{\raisebox{-0.1ex}{$\scriptstyle #1$}}
\newcommand{\lsup}[1]{\raisebox{-0.85ex}{$\scriptstyle #1$}}
\definecolor{b1}{RGB}{0,191,255}



\begin{document}
%
 % 单独编译时,其实不用编译封面目录之类的,如需要不注释这句即可
\else
\fi
%  ↓↓↓↓↓↓↓↓↓↓↓↓↓↓↓↓↓↓↓↓↓↓↓↓↓↓↓↓ 正文部分
\section{偏微分形式的热力学恒等式}
本节为补充内容,相关推导过程参考了知乎用户 \href{https://zhuanlan.zhihu.com/p/622282265}{Harogenshi} 
和 \href{https://zhuanlan.zhihu.com/p/115068655}{mosekyo} 的文章,特此致谢。

\begin{lemma}
\begin{itemize}
    偏导数之间的运算规则
    \item 倒数法则 (INV)
    \[
    \left( \frac{\partial X}{\partial Y} \right)_Z \left( \frac{\partial Y}{\partial X} \right)_Z = 1
    \]    
    \item 三元轮换法则 (TRI)
    \[
    \left( \frac{\partial X}{\partial Y} \right)_Z \left( \frac{\partial Y}{\partial Z} \right)_X \left( \frac{\partial Z}{\partial X} \right)_Y = -1
    \]
    \item 复合法则 (CP)
    \[
    \left( \frac{\partial X}{\partial Y} \right)_Z = \left( \frac{\partial X}{\partial A} \right)_B \left( \frac{\partial A}{\partial Y} \right)_Z + \left( \frac{\partial X}{\partial B} \right)_A \left( \frac{\partial B}{\partial Y} \right)_Z
    \]
    \begin{itemize}
        \item  (CP1 \( B = Z \) )
        \[
        \left( \frac{\partial X}{\partial Y} \right)_Z = \left( \frac{\partial X}{\partial A} \right)_Z \left( \frac{\partial A}{\partial Y} \right)_Z
        \]
        \item  (CP2 \( B = Y \) )
        \[
        \left( \frac{\partial X}{\partial Y} \right)_Z =  
        \left( \frac{\partial X}{\partial A} \right)_Y \left( \frac{\partial A}{\partial Y} \right)_Z
        +\left( \frac{\partial X}{\partial Y} \right)_A
        \]
    \end{itemize}
\end{itemize}
\end{lemma}
\begin{corollary}
    用大写字母表示
\(E or U\)、
\(H\)、
\(S\)、
\(A or F\)、
\(G\),用小写字母表示
\(p\)、
\(V\)、
\(T\)。
    \begin{itemize}
        \item \[
        \left(\frac{\pa b }{\pa a}\right)_Z=-\frac{1}
        {\left(\frac{\pa a}{\pa Z}\right)_b\left(\frac{\pa Z}{\pa b}\right)_a}
        =-\frac{\left(\frac{\pa Z}{\pa a}\right)_b}{\left(\frac{\pa Z}{\pa b}\right)_a}\]
        \item \[
        \left(\frac{\pa X}{\pa Y}\right)_a=\left(\frac{\pa X}{\pa b}\right)_a 
        \left(\frac{\pa b}{\pa Y}\right)_a 
        =\frac{\left(\frac{\pa X}{\pa b}\right)_a }
        {\left(\frac{\pa Y}{\pa b}\right)_a }\]
        \item \[
        \left(\frac{\pa X}{\pa a}\right)_Y =\left(\frac{\pa X}{\pa a}\right)_b 
        +\left(\frac{\pa X}{\pa b}\right)_a\left(\frac{\pa b}{\pa a}\right)_Y
        =\left(\frac{\pa X}{\pa a}\right)_b 
        +\left(\frac{\pa X}{\pa b}\right)_a
        -\frac{\left(\frac{\pa Y}{\pa a}\right)_b}{\left(\frac{\pa Y}{\pa b}\right)_a}\]
        \item 
\begin{align*}
            \left(\frac{\pa X}{\pa Y}\right)_Z 
            &=\left( \frac{\partial X}{\partial a} \right)_Z \left( \frac{\partial a}{\partial Y} \right)_Z\\
            &=\left(\left(\frac{\pa X}{\pa a}\right)_b 
            +\left(\frac{\pa X}{\pa b}\right)_a
            -\frac{\left(\frac{\pa Z}{\pa a}\right)_b}{\left(\frac{\pa Z}{\pa b}\right)_a} \right)
            \left( \frac{1}{
                \left(\frac{\pa Y}{\pa a}\right)_b 
        +\left(\frac{\pa Y}{\pa b}\right)_a
        -\frac{\left(\frac{\pa Z}{\pa a}\right)_b}{\left(\frac{\pa Z}{\pa b}\right)_a}
            }\right)
\end{align*}
            \begin{center}
                \includegraphics*[width=0.4\textwidth]{chap1/image/eq.1.jpg}
            \end{center}
            综合上述公式,我们可以把所有的偏微分关系式化成一阶的关系式
            \(\left(\frac{\pa X}{\pa a}\right)_b\)
    \end{itemize}
\end{corollary}
\begin{example}
    \(U\)的偏微分关系
    \begin{enumerate}
        \item \[\left(\frac{\pa U}{\pa T}\right)_V =C_V
        \]
        \hrule
        \item\label{1.2} \[\left(\frac{\pa U}{\pa p}\right)_V
        \overset{CP1}{=}\left(\frac{\pa U}{\pa T}\right)_V\left(\frac{\pa T}{\pa V}\right)_V
        =C_V\left(\frac{\pa T}{\pa p}\right)_V
        \]
        \hrule
        \item\label{1.3} \[\left(\frac{\pa U}{\pa T}\right)_p
        =\left(\frac{\pa H}{\pa T}\right)_p
        -p\left(\frac{\pa H}{\pa T}\right)_p
        =C_p -p\left(\frac{\pa V}{\pa T}\right)_p
        \]
        \hrule
        \item\label{1.4} \[\left(\frac{\pa U}{\pa V}\right)_p=
        \left(\frac{\pa U}{\pa T}\right)_p\left(\frac{\pa T}{\pa V}\right)_p
        \overset{\ref*{1.3}}{=}
        C_p \left(\frac{\pa T}{\pa V}\right)_p-p\left(\frac{\pa V}{\pa T}\right)_p\left(\frac{\pa T}{\pa V}\right)_p
        =C_p \left(\frac{\pa T}{\pa V}\right)_p-p
        \]
        \hrule
        \item\label{1.5} \begin{align*}
            \left(\frac{\pa U}{\pa V}\right)_T 
            &\overset{CP2}{=}\left(\frac{\pa U}{\pa V}\right)_p+
            \left(\frac{\pa U}{\pa p}\right)_V\left(\frac{\pa p}{\pa V}\right)_T\\
            &\overunderset{\ref*{1.2}}{\ref*{1.4}}{=}
            C_p\left(\frac{\pa T}{\pa V}\right)_p-p+C_V 
            \left(\frac{\pa T}{\pa p}\right)_V\left(\frac{\pa p}{\pa V}\right)_T \\
            &\overset{TRI}{=}C_p\left(\frac{\pa T}{\pa V}\right)_p-p -C_V 
            \left(\frac{\pa T}{\pa V}\right)_p \\
            &=(C_p-C_V)\left(\frac{\pa T}{\pa V}\right)_p-p
        \end{align*}
        \hrule
        \item \begin{align*}
            \left(\frac{\pa U}{\pa p}\right)_T
            &\overset{INV}{=}\left(\frac{\pa U}{\pa V}\right)_T\left(\frac{\pa V}{\pa p}\right)_T\\
            &\overunderset{\ref*{1.5}}{TRI}{=}
            -(C_p-C_V)\left(\frac{\pa T}{\pa p}\right)_V-p\left(\frac{\pa V}{\pa p}\right)_T
        \end{align*}
    \end{enumerate}
\end{example}
\begin{example}
    \( H \) 的偏微分关系
    \begin{enumerate}
        \item \[\left(\frac{\partial H}{\partial T}\right)_p = C_p \]
        \hrule
        \item\label{2.2} \[\left(\frac{\partial H}{\partial V}\right)_p 
        \overset{CP1}{=} \left(\frac{\partial H}{\partial T}\right)_p \left(\frac{\partial T}{\partial V}\right)_p 
        = C_p \left(\frac{\partial T}{\partial V}\right)_p 
        \]
        \hrule
        \item\label{2.3} \begin{align*}
        \left(\frac{\partial H}{\partial p}\right)_T 
        = \left(\frac{\partial U}{\partial p}\right)_T 
        + \left(\frac{\partial (pV)}{\partial p}\right)_T 
        &= \left(\frac{\partial U}{\partial p}\right)_T 
        + V + p\left(\frac{\partial V}{\partial p}\right)_T \\
        &=V-(C_p-C_V)\left(\frac{\pa T}{\pa p}\right)_V
        \end{align*}
        \hrule
        \item\label{2.4} \begin{align*}
            \left(\frac{\partial H}{\partial T}\right)_V 
            \overset{CP2}{=} \left(\frac{\partial H}{\partial T}\right)_p 
            + \left(\frac{\partial H}{\partial p}\right)_T \left(\frac{\partial p}{\partial T}\right)_V 
            &\overset{\ref*{2.3}}{=} C_p + \left[ V 
            + p\left(\frac{\partial V}{\partial p}\right)_T \right]
            \left(\frac{\partial p}{\partial T}\right)_V \\
            &=C_V + V\left(\frac{\pa p}{\pa T}\right)_V
        \end{align*}
        \hrule
        \item\label{2.5} \begin{align*}
            \left(\frac{\partial H}{\partial V}\right)_T 
            &\overset{CP2}{=} \left(\frac{\partial H}{\partial V}\right)_p 
            + \left(\frac{\partial H}{\partial p}\right)_V \left(\frac{\partial p}{\partial V}\right)_T \\
            &\overunderset{\ref*{2.2}}{\ref*{2.3}}{=} 
            C_p \left(\frac{\partial T}{\partial V}\right)_p 
            + \left[ V + p\left(\frac{\partial V}{\partial p}\right)_T \right] 
            \left(\frac{\partial p}{\partial V}\right)_T \\
            &\overset{TRI}{=}V\left(\frac{\partial p}{\partial V}\right)_T
            +(C_p-C_V)\left(\frac{\partial T}{\partial V}\right)_p
        \end{align*}
        \hrule
        \item \begin{align*}
            \left(\frac{\partial H}{\partial p}\right)_V 
            \overset{CP1}{=} \left(\frac{\partial H}{\partial T}\right)_V 
            \left(\frac{\partial T}{\partial p}\right)_V 
            &\overset{\ref*{2.4}}{=} 
            \left[ C_p + \left( V + p\left(\frac{\partial V}{\partial p}\right)_T \right) 
            \left(\frac{\partial p}{\partial T}\right)_V \right] 
            \left(\frac{\partial T}{\partial p}\right)_V \\
            &=C_V + V\left(\frac{\pa p}{\pa T}\right)_V
        \end{align*}
    \end{enumerate}
\end{example}
\begin{example}
    \( S \) 的偏微分关系
    \begin{enumerate}
        \item\label{3.1} \[\left(\frac{\partial S}{\partial T}\right)_V 
        =\left(\frac{\pa S}{\pa U}\right)_V\left(\frac{\pa U}{\pa T}\right)_V
        = \frac{C_V}{T} 
        \]\hrule
        \item\label{3.2} \[\left(\frac{\partial S}{\partial T}\right)_p 
        =\left(\frac{\pa S}{\pa H}\right)_p\left(\frac{\pa H}{\pa T}\right)_p
        = \frac{C_p}{T} 
        \]\hrule
        \item\label{3.3} \[\left(\frac{\partial S}{\partial V}\right)_T 
        \overset{\text{Maxwell}}{=} \left(\frac{\partial p}{\partial T}\right)_V \quad (\text{由}  dA = -S dT - p dV ) \]
        \hrule
        \item\label{3.4} \[\left(\frac{\partial S}{\partial p}\right)_T 
        \overset{\text{Maxwell}}{=} -\left(\frac{\partial V}{\partial T}\right)_p \quad (\text{由}  dG = -S dT + V dp ) \]
        \hrule
        \item\label{3.5} \begin{align*}
            \left(\frac{\partial S}{\partial V}\right)_p 
            \overset{CP1}{=} \left(\frac{\partial S}{\partial T}\right)_p \left(\frac{\partial T}{\partial V}\right)_p
            \overunderset{\ref*{3.2}}{\ref*{3.3}}{=} 
            \frac{C_p}{T} \left(\frac{\partial T}{\partial V}\right)_p  
        \end{align*}
        \hrule
        \item\label{3.6} \begin{align*}
            \left(\frac{\partial S}{\partial p}\right)_V 
            \overset{CP1}{=} \left(\frac{\partial S}{\partial T}\right)_V \left(\frac{\partial T}{\partial p}\right)_V
            \overunderset{\ref*{3.1}}{\ref*{3.4}}{=} 
            \frac{C_V}{T} \left(\frac{\partial T}{\partial p}\right)_V
        \end{align*}
    \end{enumerate}
\end{example}
\begin{example}
    \( A \) 的偏微分关系
    \begin{enumerate}
        \item\label{4.1} \[\left(\frac{\partial A}{\partial T}\right)_V = -S \quad (\text{由} \ dA = -S dT - p dV ) \]
        \hrule
        \item\label{4.2} \[\left(\frac{\partial A}{\partial V}\right)_T = -p \quad (\text{由} \ dA = -S dT - p dV ) \]
        \hrule
        \item\label{4.3} \begin{align*}
            \left(\frac{\partial A}{\partial p}\right)_V 
            \overset{CP1}{=} \left(\frac{\partial A}{\partial T}\right)_V 
            \left(\frac{\partial T}{\partial p}\right)_V 
            \overset{\ref*{4.1}}{=} -S \left(\frac{\partial T}{\partial p}\right)_V
        \end{align*}
        \hrule
        \item\label{4.4} \begin{align*}
            \left(\frac{\partial A}{\partial p}\right)_T 
            \overset{CP1}{=} \left(\frac{\partial A}{\partial V}\right)_T 
            \left(\frac{\partial V}{\partial p}\right)_T 
            \overset{\ref*{4.2}}{=} -p \left(\frac{\partial V}{\partial p}\right)_T
        \end{align*}
        \hrule
        \item\label{4.5} \begin{align*}
            \left(\frac{\partial A}{\partial T}\right)_p 
            &\overset{CP2}{=} \left(\frac{\partial A}{\partial T}\right)_V 
            + \left(\frac{\partial A}{\partial V}\right)_T 
            \left(\frac{\partial V}{\partial T}\right)_p \\
            &\overunderset{\ref*{4.1}}{\ref*{4.2}}{=} -S -p \left(\frac{\partial V}{\partial T}\right)_p
        \end{align*}
        \hrule
        \item\label{4.6} \begin{align*}
            \left(\frac{\partial A}{\partial T}\right)_p 
            &\overset{CP2}{=} \left(\frac{\partial A}{\partial T}\right)_V 
            \left(\frac{\partial T}{\partial V}\right)_p
            + \left(\frac{\partial A}{\partial V}\right)_T  \\
            &\overunderset{\ref*{4.1}}{\ref*{4.2}}{=} 
            -S\left(\frac{\partial T}{\partial V}\right)_p -p 
        \end{align*}
    \end{enumerate}
\end{example}
\begin{example}
    \( G \) 的偏微分关系
    \begin{enumerate}
        \item\label{5.1} \[\left(\frac{\partial G}{\partial T}\right)_p = -S \quad (\text{由} \ dG = -S dT + V dp ) \]
        \hrule
        \item\label{5.2} \[\left(\frac{\partial G}{\partial p}\right)_T = V \quad (\text{由} \ dG = -S dT + V dp ) \]
        \hrule
        \item\label{5.3} \begin{align*}
            \left(\frac{\partial G}{\partial V}\right)_P 
            \overset{CP1}{=} \left(\frac{\partial G}{\partial T}\right)_p 
            \left(\frac{\partial T}{\partial V}\right)_p 
            \overset{\ref*{5.1}}{=} -S \left(\frac{\partial p}{\partial V}\right)_T
        \end{align*} 
        \hrule
        \item\label{5.4} \begin{align*}
            \left(\frac{\partial G}{\partial V}\right)_T 
            \overset{CP1}{=} \left(\frac{\partial G}{\partial p}\right)_T 
            \left(\frac{\partial p}{\partial V}\right)_T 
            \overset{\ref*{5.2}}{=} V \left(\frac{\partial p}{\partial V}\right)_T
        \end{align*}
        \hrule
        \item\label{5.5} \begin{align*}
            \left(\frac{\partial G}{\partial T}\right)_V 
            &\overset{CP2}{=} \left(\frac{\partial G}{\partial T}\right)_p 
            + \left(\frac{\partial G}{\partial p}\right)_T 
            \left(\frac{\partial p}{\partial T}\right)_V \\
            &\overunderset{\ref*{5.1}}{\ref*{5.2}}{=} 
            -S + V \left(\frac{\partial p}{\partial T}\right)_V
        \end{align*}
        \hrule
        \item\label{5.6} \begin{align*}
            \left(\frac{\partial G}{\partial T}\right)_V 
            &\overset{CP2}{=} \left(\frac{\partial G}{\partial T}\right)_p 
            \left(\frac{\partial T}{\partial p}\right)_V
            + \left(\frac{\partial G}{\partial p}\right)_T \\
            &\overunderset{\ref*{5.1}}{\ref*{5.2}}{=} 
            -S\left(\frac{\partial T}{\partial p}\right)_V + V 
        \end{align*}
    \end{enumerate}
\end{example}

















%  ↑↑↑↑↑↑↑↑↑↑↑↑↑↑↑↑↑↑↑↑↑↑↑↑↑↑↑↑ 正文部分
\ifx\allfiles\undefined
\end{document}
\fi
